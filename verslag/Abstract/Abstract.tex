\section*{Abstract}
We built an extended version of 'Schelling Tipping Model' to study segregation, in which $n$ individuals of $m$ different types are placed randomly on an $l$ x $b$ board and each individual will move to a new location if less than a $q$-fraction of her neighbours share his/her type and if he/she is able to find a place that better meets this requirement. Other extensions are random displacement of the individuals and the ability to switch types.\\

In this report, we mainly studied the standard case in which $n=40$, $m=2$, $l=b=8$ and $q=\frac{1}{3}$ and we looked at the effect of type switching on the system. 
Our goal was to investigate whether such system will reach an equilibrium, and if it does, in how many generations on average (see formal definition at Introduction) and how this average (denoted as $Y$) is distributed. 
Also, we were interested in the fraction of the individual that lives in a homogenous environment (with all neighbours of his/her types) and the average number of generations it takes to reach a certain fraction of homogeneity.\\

We found that equilibrium is not always reached in the standard case (which also implies the more general case), but it is practically always reached. We saw that average of $Y$ is a non-decreasing function of $q$, and when type switching is allowed, it requires significantly less generations to reach equilibrium. We used the chi-squared test to conclude that $Y$ is not Poisson, binomially or negative binomially distributed. This is also the case when type switching is allowed. With the Kolomogorov-Smirnov test, we concluded that $Y$ is distributed differently when $q$ is varied.\\

For various values of $q$, we tested whether $60\%$, $80\%$ and full ($100\%$) segregation occured. As expected, this strongly depends on $q$. The corresponding histograms, did not have any common distribution. We found that it took on average \(3.09,3.83\) and 5.46 generations, in order to reach respectively \(60\%\) segregation, \(80\%\) segregation and \(100\%\) segregation.\\

When switching types is allowed, the average number of generations until equilibrium is reached, will drop, whereas the happiness at equilibrium will strongly increase.