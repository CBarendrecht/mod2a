% verslag mod2A
\documentclass{article}
\usepackage[utf8]{inputenc}
\usepackage{amsmath}
\usepackage{amssymb}
\usepackage{amsthm} %nodig voor blokje
\usepackage{marginnote}
\usepackage[margin=1in,footskip=0.25in]{geometry} %heb wat meer ruimte
\newcommand\header[1]{\framebox[\linewidth]{\textsc{Opgave #1}}\\}
\newcommand{\Z}{\mathbb{Z}}
\newcommand{\R}{\mathbb{R}}
\newcommand{\Q}{\mathbb{Q}}
\newcommand{\C}{\mathbb{C}}
\newcommand{\N}{\mathbb{N}}
\newcommand{\D}{\partial}

\begin{document}

\section{Average segregation time}
	As mentioned earlier, the segregation time of \(n\%\) is defined as the number of generations until at least \(n\%\) of the population on a board lives in homogenous groups. 
Where person \(i\) is said to live homogenous. If for any neighbour \(j\) of \(i\), we have \(\text{Type}(j)=\text{Type}(i)\).\\
This gives immediate rise to questions concerning the relation between the choice of \(n\) and the average segregation time at \(n\%\). 
Furthermore, it is unclear if segregation at \(n\%\) is guaranteed before a board reaches an equilibrium and what the effect is of the happiness boundary on the existence of a segregation time.\\

To research any of the given questions, we will first have to formalise our choices of board as well as the questions proposed.

\subsection{Formalisations}
The boards that will be analysed are the Standard setup and an a larger 4-Type board specified below:
\begin{table}[h!]
\centering
\caption{My caption}
\label{my-label}
\begin{tabular}{l|l|l}
  & Standard Board & 4-Type Board\\ \hline
Number of types:& 2 & 4 \\ 
 Length:& 8 & 10  \\
 Width:& 8 & 10  \\
 Happiness:& 1 & 1  \\
Population per type: & 20 & 16  
	\end{tabular}
	\end{table}
\\
The 4-Type board is constructed to maintain the same ratio of inhabited and uninhabited spots. The choice of happiness is 1 unlike the usual \(\frac{1}{3}\). 
This is to guarantee that for any \(n\leq 100\), segregation at \(n\%\) takes place prior to the board reaching equilibrium. Which leads to defining the research questions:

\begin{enumerate}
 \item ***What is the relation between the average segregation time and \(n\).***
 \item What is the average segregated fraction of the population after a board reaches equilibrium for given choices of happiness.
\end{enumerate}

\end{document}