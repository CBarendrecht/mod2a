% verslag mod2A
\documentclass{article}
\usepackage[utf8]{inputenc}
\usepackage{graphicx}
\usepackage{amsmath}
\usepackage{amssymb}
\usepackage{amsthm} %nodig voor blokje
\usepackage{marginnote}
\usepackage[margin=1in,footskip=0.25in]{geometry} %heb wat meer ruimte
\newcommand\header[1]{\framebox[\linewidth]{\textsc{Opgave #1}}\\}
\newcommand{\Z}{\mathbb{Z}}
\newcommand{\R}{\mathbb{R}}
\newcommand{\Q}{\mathbb{Q}}
\newcommand{\C}{\mathbb{C}}
\newcommand{\N}{\mathbb{N}}
\newcommand{\D}{\partial}

\begin{document}

\section{Average segregation time}
	As mentioned earlier, the segregation time of \(n\%\) is defined as the number of generations until at least \(n\%\) of the population on a board lives in homogenous groups. 
Where person \(i\) is said to live homogenous. If for any neighbour \(j\) of \(i\), we have \(\text{Type}(j)=\text{Type}(i)\).\\
This gives immediate rise to questions concerning the relation between the choice of \(n\) and the average segregation time at \(n\%\). 
Furthermore, it is unclear if segregation at \(n\%\) is guaranteed before a board reaches an equilibrium and what the effect is of the happiness boundary on the existence of a segregation time.\\
\\
To research any of the given questions, we will first have to formalise our choices of board as well as the questions proposed.\\


\subsection{Formalisations}
Prior to starting any test or properly formalising our research questions however, we note that segratation at \(n\%\) does not necessarily have to happen: 
If we consider \(n=100\) on the standard board with happiness \(1/3\). We will nearly never have total segregation before the board reaches an equilibrium.
Therefore one might instead consider the average fraction of segregation at equilibrium,for any given happiness fraction. \\
\\
Furthermore, the average segregation time as function of the segregation fraction should theoretically be a strictly increasing function since for any given board we have:
\begin{align*}
n\% \text{ lives in homogenous groups after } k \text{ generations } \Rightarrow\\
 (n-1)\% \text{ lives in homogenous groups after } k \text{ generations }
\end{align*} 
Having noted these facts, we can now properly formalise the research questions.\\
The following two questions are proposed:
\begin{enumerate}
 \item ***What is the relation between the average segregation time and \(n\).***
 \item What is the average segregated fraction of the population after a board reaches equilibrium for given choices of happiness.
\end{enumerate}

To establish results regarding these questions, we consider different setups in testings. We will be testing two different boards.
The first board to be analysed is the standard board. The second board is a larger 
"4-Type" board. The details are specified below:
\begin{table}[h!]
\centering
\caption{My caption}
\label{my-label}
\begin{tabular}{l|l|l}
  & Standard Board & 4-Type Board\\ \hline
Number of types:& 2 & 4 \\ 
 Length:& 8 & 10  \\
 Width:& 8 & 10  \\
 Happiness:& 1 & 1  \\
Population per type: & 20 & 16  
	\end{tabular}
	\end{table}
\\
The 4-Type board is constructed to maintain the same ratio of inhabited and uninhabited spots as the standard board. The choice of happiness on these boards is 1 unlike the usual \(\frac{1}{3}\). 
This guarantees that for any \(n\leq 100\), segregation at \(n\%\) takes place prior to the board reaching an equilibrium. 
To observe the average segregation time \(n\%\) for any\(n\), 500 simulations will be ran per board and averaged out in order to give an approximation for the average segregation time at \(n\%\).
Likewise the average segregated fraction will be estimate by the average of the segregated fraction of an equilibrium from 500 simulations with given happiness \(q\).
\newpage
\subsection{Results}
The results regarding the first question are shown below.

\includegraphics[scale=1]{GEN_PER_PERC_SB_1_500.fig}



\end{document}