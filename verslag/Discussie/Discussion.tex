\section{Discussion}
\subsection{Average number of generations with and without type switching}
As mentioned in the result, we could not find a proper fit for the distribution of $Y_{j}$ and $Y_{j,s}$, since it does not follow any standard discrete distribution like poisson, binomial, geometric etc. We doubt that the distributions of $Y_j$ or $Y_{j,s}$ are ever recorded in the literature. But we think that the histograms provide a good picture of the distributions. A remarkable observation we had is that the histograms, which can be defined as a random variable $h_n(x)=\frac{1}{n}\sum^n_{i=1}\mathbbm{1}_{a_{j-1}<x\leq a_j}$, for $x\in (a_{j-1},a_j]$($x$ a realisation of an random variable),  did not appear very random. What we mean here, is that when we plotted the histograms of several samples from the same $Y_j$, they are always nearly identical, which is not expected from a random variable. Although we didn't confirm this observation with a statistical test, this does indicates that $Y_j$ has a very peculiar distribution.\\

Another remark about the histograms, is that unfortunately, there is a small convention error for the number of generations(an error that we found out too late). Our model generations counters always start from 1, thus equilibrium can never be reached at the first generation by definition. This convention is actually confusing because the counter would give 2 if 1 generation has passed. So in our model, we actually assumed that one generation has always passed. But this error does not seem to affect the histogram more than just a shift of the mean.

\subsection{Average segregation time}
Like average number of generations, a distribution of the $60\%$ and $80\%$ segregation times could not be found. We should note that, when running a simulation that takes 500 instances for a given happiness rule, some uncertaincy still remains. This is best shown in figure \ref{fig:happyseg4b}, where, due to the discrete effect of the happiness rule, the average segregated fraction should be a perfect stairs pattern, but the figure shows some distortion. Of course, this is expected since 500 is 'large' but not infinite. One could discuss whether the step size for the happiness rule dependence should have been chosen larger. Since for second order neighbourhood, the number of possible happiness values is quite limited. However, due to the beforementioned inaccuracy, setting this step size too large decreases the accuracy of those figures that include happiness rule dependence.

\subsection{The Effect of switching types}
As seen in Figure \ref{fig:avesegsw}, if conformity is allowed, the population will be nearly \(80\%\) homogenous after one generations. 
This gives rise to a critical note for this extension with respect to the standard board. It is our belief namely, that the standard board is too limited in size, number of types and population density. 
As a result the standard board will reach equilibrium in no more than 2 generations on average. On extremely rare occasions, it might take 7 generations to reach equilibrium (Figure  \ref{fig:histogramSW}). 

  