\section{The effect of switching types}
\label{section:switch}
Segregation is a phenomenon taking many different forms. Rather than  limiting the scope to racial segregation, we decided to broaden our view. 
For example, one might delve into the segregation in studies with respect to friendship. 
In this case, for any person \(k\). The neighbours of \(k\) represent the \(k\)'s friends or the persons with whom \(k\) spends most of his time.
In this case, \(k\)'s type, can represent either the classes he is currently taking, or the kind of sports person \(k\) is doing, or even what kind of cellphone \(k\) has.
In any of these cases, the type of \(k\) is not necessarily set indefinitely.
In these cases, the type of \(k\) might switch, depending on the types of his friends.\\
\\
This gives rise to a new modification of the model:\\
For any person \(k\), before moving to a new location, \(k\) has a multinomial probability with to switch to a different type, where 
\[\mathbb{P}(\text{NewType}(k)=t) = \begin{cases} 
 \frac{\#\text{Neighbours of type }t}{\text{Total number of neighbours}}	&\mbox{if } \text{total number of neighbours}>0 \\ 
\mathbbm{1}_{\{t=\text{Type}(k)\}}   &\mbox{if } \text{total number of neighbours}=0
\end{cases}\]
Note that this is well defined, since for total number of neighbours \(> 0\), we have \( \mathbb{P}(\text{NewType}(k)=t) \geq 0\) and \(\sum_{i=0}^{\text{types}}\mathbb{P}(\text{NewType}(k)=t)=1\). An example of how this may effect the equilibrium, is illustrated below:
\begin{figure}[H]
	\centering
    \begin{subfigure}{0.45\textwidth}
        \includegraphics[width=\textwidth]{Voorbeeld12_12_begin}
        \caption{Situation before segregation}
        \label{fig:wis12b}
    \end{subfigure}\hspace{0cm}
    ~ 
    \begin{subfigure}{0.45\textwidth}
        \includegraphics[width=\textwidth]{Voorbeeld12_12_eind}
        \caption{Situation after segregation}
        \label{fig:wis12a}
    \end{subfigure}
    ~ 
    \caption{An illustration of the effect of switching types on the \(12 \times 12\) Board}
    \label{fig:wissel 12}
\end{figure}
A crucial difference between equilibrium with and without switching, is that if switching is allowed, the concentration of the types may differ between the starting board and the equilibrium status. 
This can be seen in figure \ref{fig:wissel 12} as the number of people of the green type decreased from 10 to 8, the yellow type even decreased from 10 to 6. 
As a matter of fact, allowing switching might lead to the extinction of several types. 
This is not generally the case for lower bounds on the happiness rule, but if the happiness rule is increased, several types might cease to exist. Despite this being quite interesting, the extinction of types is not included in this report.\\
\subsection{Average number of generations reaching equilibrium(with switch)}
\begin{figure}[H]
	\centering
    \begin{subfigure}{0.9\textwidth}
        \includegraphics[width=\textwidth]{happinessrule-totaantgenwithswitchorwithoutswitch}
    \end{subfigure}
    \caption{Number of generations until equilibrium on an 8x8 bord with standard setting, with and without type switching}
    \label{fig:AantGenS}
\end{figure}
In figure \ref{fig:AantGenS}, we see a comparison of the average number of generations for with and without type switching. There is a clear difference that when type switching is applied, it takes much less generations to reach the equilibrium. This is expected, since the probability of becoming the same type as the neighbours is more favored in our model, which results in a increase of happiness. Or, in a sociological view, the person is willing to adapt to the enviroment rather than moving on.\\
\\
From figure \ref{fig:AantGenS}, we observed that the average generations is nearly the same for HR 0.6 til 1. Just to get an idea whether this statement is justified, we selected only three HR's, namely 0.6, 0.8 and 1 and performed the two sample t-test(ttest2) in Matlab. This t-test tests the null-hypothesis $H_0:\mu=\nu$ with $\mu,\nu$ respectively the mean of the distribution of $X$ and $Y$, against $H_1:\mu\neq\nu$. It uses the test statistics:
 \[T=\frac{\overline{X}-\overline{Y}}{\sqrt{\frac{S^2_X}{m}-\frac{S^2_Y}{n}}}\]
This test assumes that the mean difference $\mu-\nu$ is normally distributed, and since our data are from the discrete random variables, this assumption would not make sense. However, according to the central limit theorems, we may make this assumption for large enough sample size. To avoid difficult calculations of how big the sample size is needed, we simply chose a size of 50000. Also, in our case, we made sure $m=n$, and the test can also be adjusted to not assume equal variance.\\
\\
After the simulations, we obtained the following sample means:
\begin{table}[htp]
\centering
\caption{Sample means of number of generations simulated with HR 0.6, 0.8 and 1, with 50000 sample size for each HR.}
\begin{tabular}{|l|l|l|l|}
\hline
 HR&0.6&0.8&1 \\ \hline
 Sample Mean&2.5213&2.5616&2.5711  \\ \hline 
\end{tabular}
\end{table}
We applied the t-test two times. First time comparing the mean of HR 0.6 and 0.8. Second time for comparing the mean of 0.8 and 1. The first test rejected the null-hypothesis, the second didn't. So with 95\% confidence level, we may conclude that the average generations are unequal for HR 0.6 vs 0.8, but equal for HR 0.8 vs 1.