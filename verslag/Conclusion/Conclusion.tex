\section{Conclusion}
\subsection{Equilibrium is not always reached}
Coming back to the first research questions, we have seen that equilibrium is not always reached on the standard board in finite time. This is because a periodic movement of individuals may occur. Also, equilibrium might not always be reached on a larger board (with equivalent settings), since we can just implement this periodic solution on the larger board.\\

\subsection{Average number of generations until equilibrium}
For the standard model, we have seen that equilibrium is reached on average at the third generation, and that the average is a non-decreasing function of the happiness rule in the model. When type switching is applied, the equilibrium is reached on average at the second generation for the standard model. And again, the average is non-decreasing vs the happiness rule. We have also observed that type switching strongly stimulates segregation.\\

We have concluded that $Y_j$ and $Y_{j,s}$, which respectively denotes the number of generations it takes to reach the equilibrium without or with type switching, is neither Poisson, Binomially, Geometrically, or Negative Binomially distributed, for $j\in I=\{1/4, 1/3, 1/2 ,1\}$. The QQ-plot suggests that $Y_j$ and $Y_k$, for $j\neq k$, $j,k\in I$, might come from the location-scale family of some distribution however. This could not be confirmed by a statistical test, instead, with the Kolmogorov-Smirnov test, it was concluded that $Y_j$ and $Y_k$ are not distributed with the same parameter. Furthermore, the QQ-plot appeared linear for $Y_1$ and $Y_{1,s}$.\\

\subsection{Average segregation time}

\subsubsection*{Segregated fraction}
Coming back to the third research question: "How is this segregation time distributed with a Happiness Rule of \(1\) and how does the Happiness Rule affect the average segregation time?".\\
We conclude that the average segregation time  is 3.09 at \(60\%\), 3.83 at \(80\%\) and 5.46 at \(100\%\).
We have seen that the relation of the average segregation time depending on the segregated fraction considered, is neither linear nor exponential. For the standard board, it takes three times as long to reach $50\%$ segregation than $40\%$. After that barrier has been taken, segregation takes place relatively easily. On the standard board, full segregation is, on average, reached after 5.4 generations. For the 4-Type board this number is 8.5.

\subsubsection*{Forming of Homogenous Groups}
On the standard board, $60\%$ segregation is almost always reached when a Happiness Rule $>0.6$ is chosen. If we increase to $80\%$, the Happiness Rule must be greater than 0.7. For full segregation, a HR of 0.8 is required. On the 4-Type board, these values lie slightly higher, except for full segregation, which is not guaranteed on the 4-Type board.

\subsubsection*{60\% and 80\% segregation times}
We found that on average, for the standard board and for Happiness Rule 1, it took about 3.09 generations to reach $60\%$ segregation. 3.83 generations for $80\%$. The distribution of these segregation times was hard to determine, but the earlier mentioned averages can be considered statistically different.

\subsection{The Effect of switching types}
The third research question stated: "If an individual is able to switch to another type with a probability that depends on the types of his/her neighbours, how does this affect the average number of generations (until equilibrium is reached) and to what extent does it impact the corresponding distribution and the other questions posed earlier?"

From Figure \ref{fig:AantGenS} it can clearly be concluded that it will take at most 2.5 generations on average until equilibrium is reached if it is possible to switch types. Whereas it may take up to 5.5 generations to reach equilibrium, if switching is not allowed. We also conclude that the corresponding histograms (with and without switching),do not share the same distribution.
We see a strong increase in the average happiness at equilibrium when switching is allowed and the average segregation time never exceeds 2 generations for any \(p<80\%\).