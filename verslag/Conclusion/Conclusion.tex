\section{Conclusion}
\subsection{Equilibrium is not always reached}
Coming back to the research questions, we've seen that equilibrium is not always reached in the standard model. Because a periodic movement of individuals may occur. Also, equilibrium is not always reached on a larger board(with still standard settings for other parameters), since we can just implement this periodic solution on the larger board.\\

\subsection{Average number of generations until equilibrium}
For the standard model, we've seen that equilibrium is reached on average at the 3rd generation, and that the average is a non-decreasing function of the happiness rule in the model. When type switching is applied, the equilibrium is reached on average at the 2nd generation for the standard model. And again, the average is non-decreasing vs the happiness rule. We've also observed that type switching strongly stimulates segregation.\\

We've concluded that $Y_j$ and $Y_{j,s}$, which respectively denotes the number of generations it takes to reach the equilibrium without or with type switching, is not poisson, binomial, geometrical, and negative binomial distributed, for $j\in I=\{1/4, 1/3, 1/2 ,1\}$. But the QQ-plot suggests that $Y_j$ and $Y_k$, for $j\neq k$, $j,k\in I$, might come from the location-scale family of some distribution. This could not be confirmed by a statistical test, instead, with the kolmogorov-smirnov test, it was concluded that $Y_j$ and $Y_k$ are not distributed with the same parameter. Furthermore, the QQ-plot appeared linear for $Y_1$ and $Y_{1,s}$.\\

\subsection{Average segregation time}

\subsubsection*{Segregated fraction}
We've seen that for the average segregation time depending on the segregated fraction considered, the relation is neither linear nor exponential. For the standard board, it takes three times as long to reach $50\%$ segregation than $40\%$. After that barrier has been taken, segregation takes place relatively easily. On the standard board, full segregation is, on average, reached after 5.5 generations. For the 4-Type board this number is 8.5.

\subsubsection*{Forming of Homogenous Groups}
On the standard board, $60\%$ segregation is almost always reached when happiness rule $>0.6$. If we increase to $80\%$, the happiness rule must be greater than 0.7. For full segregation, a HR of 0.8 is required. On the 4-Type board, these values lie slightly higher, exept for the full segregation, which is not guaranteed on the 4-Type board.

\subsubsection*{60\% and 80\% segregation times}
We found that on average, for the standard board and for Happiness Rule 1, it took about 2.09 generations to reach $60\%$ segregation. 2.83 generations for $80\%$. The distribution of these segregation times was hard to determine, but the beforementioned averages can be considered statistically different.