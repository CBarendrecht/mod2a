\section{Conclusion}
Coming back to the research questions, we've seen that equilibrium is not always reached in the standard model. Because a periodic movement of individuals may occur. Also, equilibrium is not always reached on a larger board(with still standard settings for other parameters), since we can just implement this periodic solution on the larger board.\\
\\
For the standard model, we've seen that equilibrium is reached on average at the 3rd generation, and that the average is a non-decreasing function of the happiness rule in the model. When type switching is applied, the equilibrium is reached on average at the 2nd generation for the standard model. And again, the average is non-decreasing vs the happiness rule. We've also observed that type switching strongly stimulates segregation.\\
\\
We've concluded that $Y_j$ and $Y_{j,s}$, which respectively denotes the number of generations it takes to reach the equilibrium without or with type switching, is not poisson, binomial, geometrical, and negative binomial distributed, for $j\in I=\{1/4, 1/3, 1/2 ,1\}$. But the QQ-plot suggests that $Y_j$ and $Y_k$, for $j\neq k$, $j,k\in I$, might come from the location-scale family of some distribution. This could not be confirmed by a statistical test, instead, with the kolmogorov-smirnov test, it was concluded that $Y_j$ and $Y_k$ are not distributed with the same parameter. Furthermore, the QQ-plot appeared linear for $Y_1$ and $Y_{1,s}$.