% verslag mod2A
\documentclass{article}
\usepackage[utf8]{inputenc}
\usepackage{amsmath}
\usepackage{amssymb}
\usepackage{amsthm} %nodig voor blokje
\usepackage{marginnote}
\usepackage[margin=1in,footskip=0.25in]{geometry} %heb wat meer ruimte
\newcommand\header[1]{\framebox[\linewidth]{\textsc{Opgave #1}}\\}
\newcommand{\Z}{\mathbb{Z}}
\newcommand{\R}{\mathbb{R}}
\newcommand{\Q}{\mathbb{Q}}
\newcommand{\C}{\mathbb{C}}
\newcommand{\N}{\mathbb{N}}
\newcommand{\D}{\partial}

\title{Segregation project 2017}
\author{Casper Barendrecht, Guanyu Jin, Stijn Moerman, Nand Snijder}
\date{6 April 2017}

\begin{document}
\reversemarginpar 
\maketitle
<<<<<<< HEAD
\section{Abstract}
An extended version of the 'Shelling Tipping Model' is built...
\section{Introduction}
In 1978, Thomas C Schelling developed his tipping model by placing pennies and dimes on a chess bord and moved them according to various rules. By viewing the pennies and dimes as two types of people, the rule of moving as a preference for the individuals, and the chess bord as a city, he soon discovered that segragation is formed on the bord, even when the preference of the individuals is very subtle.\\
\\
Based on this idea, we first built a basic model which consists of a 8x8 bord with 40 individuals that are divided in 2 types. The individuals are moved according to their 'Happiness' in the current place. For the basic model, an individual is considered happy if $\frac{1}{3}$ of his/her second order neighbours(for example a person not on the edge can have 8 neighbours) is of the same type. Otherwise, an individual is considered unhappy and will be moved to the nearest place such that his/her happiness is strictly higher, which we referred to as the 'Happiness Rule'. After that, we extended the basic model by changing the parameters such as the size of the population, bord and number of types. We also included an option for random displacement: an individual which is not happy will be moved to a empty location randomly.(Maybe more extension...)\\\\
For both the basic and the extended model, we ran 500 simulations several times and investigated how different values of the parameters affected the segregation pattern. In order to formulate our research goals precisely, the following definitions are important:\\
\\
1.\textbf{Generation}: A population is said to have entered a next generation if the happiness of every person has been checked once. \\
2.\textbf{Equilibrium}: The population is said to have reached an equilibrium if zero individual is moved after a generation\\
3.\textbf{Segregation time of n$\%$}: The segregation time of n$\%$ is defined as the number of generations such that n$\%$ of the population has all his/her neighbours of the same type.
\\
\\
For this project, we focussed on the following main questions:\\
1. How does the parameters affect the equilibrium? Does the population always reach an equilibrium? How many generations on average does it take to reach an equilibrium? What's the probability distribution of the number of generations to reach an equilibrium?\\
2. What fraction of the individuals is happy after the equilibrium? Can we optimize that by changing the size of the bord?\\
3. What is the segregation time for 60$\%$ and how does the definition of the happiness affect it?\\
\\
Short abstract about main findings...\\
how the report looks like further on...
\section{Proof of equilibrium}

\marginnote{Theorem}
For an $8\times 8$ board with 20 characters of type 1 and 20 characters of type 2, happiness rule of $1/3$, and displacement to the nearest spot with greater happiness (if it exists), an equilibrium will always be reached.
\begin{proof}
Let an $8\times 8$ board be given and randomly placed individuals. If there is equilibrium, there is nothing to prove. So assume there is no equilibrium. Then there is an individual $i$ with happiness$h := (x_i,y_i,\text{type}(i)) < 1/3$ and there is a spot $(x,y)\in X$ with happiness$(x,y,type(i)) > h$.
\end{proof} 



\end{document}