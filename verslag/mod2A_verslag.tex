% verslag mod2A
\documentclass{article}
\usepackage[utf8]{inputenc}
\usepackage{amsmath}
\usepackage{amssymb}
\usepackage{amsthm} %nodig voor blokje
\usepackage{marginnote}
\usepackage[margin=1in,footskip=0.25in]{geometry} %heb wat meer ruimte
\newcommand\header[1]{\framebox[\linewidth]{\textsc{Opgave #1}}\\}
\newcommand{\Z}{\mathbb{Z}}
\newcommand{\R}{\mathbb{R}}
\newcommand{\Q}{\mathbb{Q}}
\newcommand{\C}{\mathbb{C}}
\newcommand{\N}{\mathbb{N}}
\newcommand{\D}{\partial}

\title{Segregation project 2017}
\author{Casper Barendrecht, Guanyu Jin, Stijn Moerman, Nand Snijder}
\date{6 April 2017}

\begin{document}
\reversemarginpar 
\maketitle
\section{Introducion}

\section{Definitions}
\marginnote{Generation}
A \underline{generation} is a sequence of turns in which every individual is selected once.\\
\marginnote{Equilibrium}
A board $X$ has reached \underline{equilibrium} after $n$ generations if, in the $n+1$th generation, no one has moved.

\section{Proof of equilibrium}

\marginnote{Theorem}
For an $8\times 8$ board with 20 characters of type 1 and 20 characters of type 2, happiness rule of $1/3$, and displacement to the nearest spot with greater happiness (if it exists), an equilibrium will always be reached.
\begin{proof}
Let an $8\times 8$ board be given and randomly placed individuals. If there is equilibrium, there is nothing to prove. So assume there is no equilibrium. Then there is an individual $i$ with happiness$h := (x_i,y_i,\text{type}(i)) < 1/3$ and there is a spot $(x,y)\in X$ with happiness$(x,y,type(i)) > h$.
\end{proof} 


\end{document}