% verslag mod2A
\documentclass{article}
\usepackage[utf8]{inputenc}
\usepackage{amsmath}
\usepackage{amssymb}
\usepackage{amsthm} %nodig voor blokje
\usepackage{marginnote}
\usepackage[margin=1in,footskip=0.25in]{geometry} %heb wat meer ruimte
\newcommand\header[1]{\framebox[\linewidth]{\textsc{Opgave #1}}\\}
\newcommand{\Z}{\mathbb{Z}}
\newcommand{\R}{\mathbb{R}}
\newcommand{\Q}{\mathbb{Q}}
\newcommand{\C}{\mathbb{C}}
\newcommand{\N}{\mathbb{N}}
\newcommand{\D}{\partial}

\title{Segregation project 2017}
\author{Casper Barendrecht, Guanyu Jin, Stijn Moerman, Nand Snijder}
\date{6 April 2017}

\begin{document}
\reversemarginpar 
\maketitle
<<<<<<< HEAD
\section{Introduction}
In 1978, Thomas C Schelling developed his tipping model by placing pennies and dimes on a chess bord and moved them according to various rules. By viewing the pennies and dimes as two types of people, the rule of moving as a preference for the individuals, and the chess bord as a city, he soon discovered that segragation is formed on the bord, even when the preference of the individuals is very subtle.\\
\\
In order to gain more insight in the pattern formation of a population when each individuals are having the same motives of moving, we first built a basic model based on Schelling's idea, and extended it by increasing the size of the population, bord and number of types,etc. In this project, we investigated how the size of the bord, populations and amount of types affects the segregation pattern, when certain rule of moving, which is referred to as the 'happiness rule', is applied. To formulate the research question more formally, the following definitions are made:\\
1.\textbf{Neighbouhrhood}:Given a person on the m x n bord. The neighbours of a person are the people... \\
2.\textbf{Happiness rule}: A person is happy if a fraction(which is $1/3$ in the basic model) of his/her neighbourhood is of the same type as him/her.\\
3.\textbf{Generation}: A population is said to reach a next generation if every person has.. \\
4.\textbf{Equilibrium}: The population is said to reach equilibrium if no person is moved after a generation.\\
\\
The research questions are:\\
1. Given a 8x8 bord, 2 types and 40 people, will the population ever reach the equilibrium? And what will happen if we increase the population, types or size of the bord?\\
2. After 500 simulations, 
=======
\section{Introducion}

\section{Definitions}
\marginnote{Generation}
A \underline{generation} is a sequence of turns in which every individual is selected once.\\
\marginnote{Equilibrium}
A board $X$ has reached \underline{equilibrium} after $n$ generations if, in the $n+1$th generation, no one has moved.

\section{Proof of equilibrium}

\marginnote{Theorem}
For an $8\times 8$ board with 20 characters of type 1 and 20 characters of type 2, happiness rule of $1/3$, and displacement to the nearest spot with greater happiness (if it exists), an equilibrium will always be reached.
\begin{proof}
Let an $8\times 8$ board be given and randomly placed individuals. If there is equilibrium, there is nothing to prove. So assume there is no equilibrium. Then there is an individual $i$ with happiness$h := (x_i,y_i,\text{type}(i)) < 1/3$ and there is a spot $(x,y)\in X$ with happiness$(x,y,type(i)) > h$.
\end{proof} 
>>>>>>> f7fbb67d673c216f8057b593c32f5f9f64f8acea


\end{document}