% verslag mod2A
\documentclass{article}
\usepackage[utf8]{inputenc}
\usepackage{float}
\usepackage{amsmath}
\usepackage{amssymb}
\usepackage{amsthm} %nodig voor blokje
\usepackage{marginnote}
\usepackage[outdir=./]{epstopdf}
\usepackage[margin=1in,footskip=0.25in]{geometry} %heb wat meer ruimte
\usepackage{graphicx}
\newcommand{\Z}{\mathbb{Z}}
\newcommand{\R}{\mathbb{R}}
\newcommand{\Q}{\mathbb{Q}}
\newcommand{\C}{\mathbb{C}}
\newcommand{\N}{\mathbb{N}}
\newcommand{\D}{\partial}

\title{Segregation}
\author{Casper Barendrecht, Guanyu Jin, Stijn Moerman, Nand Snijder}
\date{6 April 2017}

\begin{document}
\reversemarginpar 
\maketitle
\section{Abstract}
An extended version of the 'Shelling Tipping Model' is built...
\section{Introduction}
In 1978, Thomas C. Schelling developed his tipping model by placing pennies and dimes on a chess board and moved them according to various rules. By viewing the pennies and dimes as two types of people, the rule of moving as a preference for the individuals, and the chess board as a city, he soon discovered that segregation is formed on the board, even when the preference of the individuals is very subtle.\\
\\
Based on this idea, we first built a basic model which consists of an $8\times8$ board with 40 individuals that are divided in two types. The individuals are moved according to their 'Happiness' in the current place. For the basic model, an individual is considered happy if $\frac{1}{3}$ of his/her second order neighbours (a person has at most 8 neighbours) is of the same type. Otherwise, an individual is considered unhappy and will be moved to the nearest place such that his/her happiness is strictly higher, which will be referred to as the 'Happiness Rule'. After that, we extended the basic model by changing the parameters such as the size of the population, board and number of types. We also included an option for random displacement: an individual which is not happy will be moved to a empty location randomly. (That is to say, that individual will be placed to any empty spot with equal probability.)\\\\
For both the basic and the extended model, we ran 500 simulations several times and investigated how different values of the parameters affected the segregation pattern. In order to formulate our research goals precisely, the following definitions are important:\\
\\
1. \textbf{Generation}: A population is said to have entered a next generation if the happiness of every individual has been checked once. \\
2. \textbf{Equilibrium}: The population is said to have reached an equilibrium if there are no individuals that have moved in the past generation.\\
3. \textbf{Segregation time at n$\%$}: The segregation time at n$\%$ is defined as the number of generations such that n$\%$ of the population has all his/her neighbours of the same type.
\\
\\
For this project, we focussed on the following main questions:\\
1. How do the parameters affect the equilibrium? Does the population always reach an equilibrium? How many generations on average does it take to reach an equilibrium? What's the probability distribution of the number of generations to reach an equilibrium?\\
2. What fraction of the individuals is happy after the equilibrium? Can we optimize that by changing the size of the bord?\\
3. What is the segregation time for 60$\%$ and how does the definition of the happiness affect it?\\
\\
Short abstract about main findings...\\
how the report looks like further on...
\input{Equilibrium}


\section{Average segregation time}
	As mentioned earlier, the segregation time of \(n\%\) is defined as the number of generations until at least \(n\%\) of the population on a board lives in homogenous groups. 
Where person \(i\) is said to live homogenous if for any neighbour \(j\) of \(i\), we have \(\text{Type}(j)=\text{Type}(i)\).\\
This gives immediate rise to questions concerning the relation between the choice of \(n\) and the average segregation time at \(n\%\). 
Furthermore, it is unclear if segregation at \(n\%\) is guaranteed before a board reaches an equilibrium and what the effect is of the happiness boundary on the existence of a segregation time.\\
\\
To research any of the given questions, we will first have to formalise our choices of board as well as the questions proposed.\\


\subsection{Formalisations}
Prior to starting any test or properly formalising our research questions however, we note that segratation at \(n\%\) does not necessarily have to happen: 
If we consider \(n=100\) on the standard board with happiness \(1/3\). We will nearly never have total segregation before the board reaches an equilibrium.
Therefore one might instead consider the average fraction of segregation at equilibrium,for any given happiness fraction. \\
\\
Furthermore, the average segregation time as function of the segregation fraction should theoretically be a strictly increasing function since for any given board we have:
\begin{align*}
&n\% \text{ lives in homogenous groups after } k \text{ generations } \Rightarrow\\
& m\% \text{ lives in homogenous groups after } k \text{ generations, for any } 0 \leq m \leq n
\end{align*} 
Having noted these facts, we can now properly formalise the research questions.\\
The following questions are proposed:
\begin{enumerate}
 \item What is the relation between the average segregation time and \(n\).
 \item What is the average segregated fraction of the population after a board reaches equilibrium for given choices of happiness.
 \item For any Happiness Rule from 0 to 1, how often do we reach at least $60\%$, at least $80\%$ or even full ($100\%$) segregation?
 \item For $60\%$ and $80\%$ segregation, what distribution do we get for the segregation time, if $HR = 1$? Are these comparable?
\end{enumerate}

To establish results regarding these questions, we consider different setups in testings. We will be testing two different boards.
The first board to be analysed is the standard board. The second board is a larger 
"4-Type" board. The details are specified below:
\begin{table}[h!]
\centering
\caption{Specs of the two considered boards}
\label{my-label}
\begin{tabular}{l|l|l}
  & Standard Board & 4-Type Board\\ \hline
Number of types:& 2 & 4 \\ 
 Length:& 8 & 10  \\
 Width:& 8 & 10  \\
 Happiness:& 1 & 1  \\
Population per type: & 20 & 16  
	\end{tabular}
	\end{table}
\\
The 4-Type board is constructed to maintain the same ratio of inhabited and uninhabited spots as the standard board. The choice of happiness on these boards is 1 unlike the usual \(\frac{1}{3}\). 
This guarantees that for any \(n\leq 100\), segregation at \(n\%\) takes place prior to the board reaching an equilibrium. 
To observe the average segregation time \(n\%\) for any \(n\), 500 simulations will be ran per board and averaged out in order to give an approximation for the average segregation time at \(n\%\).
Likewise the average segregated fraction will be estimate by the average of the segregated fraction of an equilibrium from 500 simulations with given happiness \(q\).
\newpage
\subsection{Results}
\subsubsection{Question 1}
The results regarding the first question are shown below:\\
\begin{figure}[H]
    \centering
    \includegraphics[width=0.8\textwidth]{aveseg_sb_2}
    \caption{Average segregation time on the standard board}
    \label{fig:avesegsb}
\end{figure}

Most notable of figure \ref{fig:avesegsb} is that it is neither linear nor exponential, which is what one might initially expect. Instead, it appears to be partially exponential and partially lineair. From the figure, we note that the average segregation time increases fastest between 0.4 and 0.5. 
Which is to say that it takes three times longer for 50\% of the population to live in homogenous groups than for 40\%.\\
Also note that the 'lift off' is approximately at $\frac{1}{3}$, but it is unlikely that this has anything to do with the standard happiness rule of $\frac{1}{3}$.\\
\textbf{Partial explanation.} The first part of the graph is easy to understand. Because of the initial random placement of the individuals, the odds are quite good that few individuals are already homogenous.\\
For the second part between 0.4 and 0.5, it appears to take a relatively long time to transform from the initial chaos to $50\%$ segregation. A 'boundary' has to be taken, comparable with the activation energy for chemical reactions.\\
If 50$\%$ segregation has been reached, it becomes more easy to segregate even further, because individuals can easily move from one homogenous group (of the other type) to another (of their own type), which explains the part between $50\%$ and $70\%$. One'd expect this to continue and it'd be done in another one or two generations, but with $70\%$ segregation, it gets harder for individuals to move to a location that is homogenous.\\
\\
Next, we consider the same question but this time for the 4-Type board, and we get the following result.


 \begin{figure}[H]
    \centering
    \includegraphics[width=0.8\textwidth]{aveseg_4b_2}
    \caption{Average segregation time on the 4-Type board}
    \label{fig:aveseg4b}
\end{figure}

The most notable difference between this figure and the figure \ref{fig:avesegsb} is that the 'lift off' lies much earlier than in figure \ref{fig:avesegsb}. Now based on the number of types, you'd expect this to be around half as soon.\\\\
\textbf{Now why is the fraction at which figure \ref{fig:aveseg4b} lifts off not exactly half, but sooner than the lift off of figure \ref{fig:avesegsb}?}
Although this difference feels intuitive, we will give a formal explanation. Consider the probability that any individual $i$ has $a$ neighbours of the same type, and no neighbours of another type, given $k-a$ empty spots, where $k$ is the maximum number of possible neighbours for the location of $i$. In the basic model, this probability equals
\[
p_b = \frac{19}{39}\frac{18}{38}...\frac{19-a}{39-a} = \frac{\frac{19!}{(19-a)!}}{\frac{39!}{(39-a)!}} = \frac{19!(39-a)!}{39!(19-a)!}
\]
In the 4-Type model, this probability is
\[
p_4 = \frac{15!(63-a)!}{63!(15-a)!}
\]
which is less than half of $p_b$.\\
Also note that an individual is more likely to be homogenous if it is placed at a corner or edge. This is because a corner or edge spot has less possible neighbours than an interior spot, thus increasing the probability of only having neighbours of your own type. The 4-Type board has relatively fewer edge positions in comparison to the standard board. This effect also decreases the fraction of segregation prior to displacement of individuals.
However the people/boardspace ratio is almost the same in both models and thus should not have too much of an impact on the results.\\
\\
Another effect is the rise of the number of generations, that is, the time it takes before the board reaches full segregation. However, this effect is less strong than one might initially expect. On the standard board, it takes on average 5.5 generations to reach full segregation. On the 4-Type board however, it takes 8.5.\\
\\
The remaining part of figure \ref{fig:aveseg4b} is quite similar to figure \ref{fig:avesegsb}: quick rise, followed by a slightly decreasing ascending relation. The explanation is similar to that of the basic model.
\subsubsection{Question 2}
\begin{figure}[H]
    \centering
    \includegraphics[width=0.8\textwidth]{habysegfrac_sb_2}
    \caption{Average segregated fraction as a result of happiness on the standard board}
    \label{fig:happysegsb}
\end{figure}

Figure \ref{fig:happysegsb} displays three scattered values. The red and green dots represent the minimum and maximum segregrated fraction of 500 boards for the given happiness. Notable in this picture is that the average segregated value is greater than the happiness. 
Another thing worth noting is that the segregated fractions tend to appear in different groups seperated by relatively large percentages.
\begin{figure}[H]
    \centering
    \includegraphics[width=0.8\textwidth]{habysegfrac_4b_2}
    \caption{Average segregated fraction as a result of happiness on the 4-Type board}
    \label{fig:happyseg4b}
\end{figure}
Figure \ref{fig:happyseg4b} shows that contrast to the standard board, this board does not respect the property that the average segregated fraction is always greater than or equal to the happiness. 
A thing of interest however, is that the averages of both figure \ref{fig:happysegsb} and \ref{fig:happyseg4b} seem to cluster. This is because, as mentioned earlier, the effect of the Happiness Rule is quite discrete when using second order neighbourhood.
\subsubsection{Question 3}
In order to answer the third question, we ran 500 simulations for each HR ranging from 0 to 1 with increment 0.025, and for each segregation $\%$, 60, 80 and 100. The results are shown below.


\subsection{type of neighborhood}



\end{document}