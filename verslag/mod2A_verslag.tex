% verslag mod2A
\documentclass{article}
%table of contents
\usepackage[utf8]{inputenc}

%meer ams meer beter

\usepackage{amsmath}
\usepackage{amssymb}
\usepackage{amsthm} %nodig voor blokje
\usepackage{bbm} %indicator

\usepackage{marginnote}
\usepackage[margin=1in,footskip=0.25in]{geometry} %heb wat meer ruimte

%Plaatjes
\usepackage{graphicx}
\usepackage{float}
\usepackage{caption}
\usepackage{subcaption}

\usepackage{import} % handig bij importeren van dingen uit andere mapjes
\usepackage{wallpaper} %voorblad

\usepackage{wrapfig} %plaatjes in tekst

\newcommand*{\blankpage}{%
\vspace*{\fill}
{\centering This page is left intentionally blank.\par}
\vspace{\fill}}


\begin{document}
\import{./Voorblad/}{Voorblad}
\newpage


\section*{Abstract}
We built an extended version of 'Schelling Tipping Model' to study segregation, in which $n$ individuals of $m$ different types are placed randomly on an $l$ x $b$ board and each individual is moved when she has less than $p$-fraction of her neighbours that are her type and is able to find a place that better meets this requirement. Other extensions are random displacement of the individuals and ability to switch types.\\
\\
In this report, we mainly studied the standard case in which $n=40$, $m=2$, $l=b=8$ and $p=\frac{1}{3}$ and we also looked at the effect of type switching on the system. Our goal is to investigate whether such system will reach an equilibrium, and if it does, in how many generations on average(see formal definition at Introduction) and how this average (denoted as $Y$) is distributed. Also, we are interested in the fraction of the individual that lives in a homogenous environment(with all neightbours of her types) and the average number of generations in takes to reach certain fraction of homogeneity.\\
\\
We found out that equilibrium is not always reached in the standard case(which also implies the more general case), but it is practically always reached. We saw that average of $Y$ is a non-decreasing function of $p$, and when type switching is applied, the equilibrium requires significantly less generations. We used the chi-sqaured test to conclude that $Y$ is not poisson, binomial and negative binomial distributed. This is also the case when type switching is possible. With the kolomogorov-smirnov test, we concluded that $Y$ is distributed differently when $p$ is varied.


\newpage

\tableofcontents
\newpage

\import{./Introduction/}{Introduction}
\newpage

\import{./Model/}{Model}
\newpage

\input{Equilibrium}
\newpage

\import{./VerslagdeelAantgen/}{VerslagdeelAantgen}
\newpage

\import{./happiness_01/}{VerslagHappinessgedeelte}
\newpage

\import{./AverageSeg/}{AveSegtime}
\newpage

\import{./Switch/}{Switch}
\newpage

\import{./Conclusion/}{Conclusion}
\newpage

\import{./Discussie/}{Discussion}
\newpage

\addcontentsline{toc}{section}{Appendix}
\import{./Appendix/}{Appendix}

\end{document}