\chapter{Conclusie}
We onderzochten hoe we het proces van een rivier oversteken met een bootje kunnen optimaliseren. We hebben de hamiltoniaanstelling gebruikt om ons model te vertalen naar een stelsel van differentiaalvergelijkingen. Allereerst hebben we dit gedaan voor een boot met constante snelheid, waarbij we de oversteektijd probeerden te minimaliseren. Voor een constant stromingsprofiel was dit analytisch op te lossen en dit leverde verwachte resultaten op: netto rechtdoor varen levert de kortste tijd op. Wanneer het stromingsprofiel complexere vormen aan begon te nemen was analytisch oplossen niet meer mogelijk, maar leverde analytisch vereenvoudigen en oplossen met MATLAB betrouwbare resultaten.

Vervolgens hebben we ons model uitgebreid om ook met energie en snelheid te kunnen rekenen. Ook problemen van deze soort hebben we eerst gedeeltelijk analytisch opgelost en daarna met MATLAB uitgerekend. We bevinden de hamiltoniaanstelling in zijn analytische context gecombineerd met MATLAB een goed gereedschap om te berekenen hoe je op optimale wijze een boot naar de overkant van een rivier stuurt.
