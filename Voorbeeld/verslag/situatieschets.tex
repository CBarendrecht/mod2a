\chapter{Situatieschets}\label{sec:situatieschets}

\begin{wrapfigure}{r}{4.5cm}
	\usetikzlibrary{calc}
\begin{tikzpicture}[scale = 3.5, transform shape]
	% Variables
	\newcommand{\boatAngle}{25}
	\newcommand{\vectorLength}{.25}
	\newcommand{\textSize}{.15}

	% River banks
	\draw (0, 1) -- (0, -1);
	\draw (1, 1) -- (1, -1);
	
	% Coordinate system
	\draw[->, thick] (0, 0) -- (1, 0);
	\draw[->, thick] (0, 0) -- (0, 1);
	
	% Coordinate system labels. The scope structures are used to make the text smaller.
	\begin{scope}[shift={(1, 0)}]
		\begin{scope}[scale=\textSize * 1.5, transform shape]
			\node[anchor=north east, font=\fontsize{10}{3}] at (0, 0) {$ x_1 $};
		\end{scope}
	\end{scope}
	\begin{scope}[shift={(0, 1)}]
		\begin{scope}[scale=\textSize * 1.5, transform shape]
			\node[anchor=north east, font=\fontsize{10}{3}] at (0, 0) {$ x_2 $};
		\end{scope}
	\end{scope}
	
	% Vector field
	\newcommand{\xVectors}{5} % Number of vectors in the x_1 direction
	\newcommand{\yVectors}{3} % Number of vectors in the x_2 direction
	\FPeval\xVectorsr{\xVectors - .5} % Each x_1 vector lies exactly between two integers
	\FPeval\yVectorsStart{1 - \yVectors} % The first y vector has to be one higher in order to have the lowest vectors inside the picture. 
	\foreach \x in {.5, ..., \xVectorsr} {
		\foreach \y in {\yVectorsStart, ..., \yVectors} {
			\FPeval\vectorLength{sin(\x/\xVectors * pi) / \yVectors / 2} % The length of the vector at \x, \y, calculated with a sine function.
			\draw[->, draw=gray!75!white, very thin] (\x/\xVectors, \y/\yVectors) -- (\x/\xVectors, \y/\yVectors - \vectorLength);
		}
	}
	
	% s(x_1) label
	\begin{scope}[shift={(.5, .9)}]
		\begin{scope}[scale=\textSize, transform shape]
			\node[anchor=west, font=\fontsize{10}{3}] at (0, 0) {$ s(x_1) $};
		\end{scope}
	\end{scope}
	
	% Boat center coordinate
	\coordinate (c) at (.3, .1);
	
	% Dashed boat path
	\draw[draw=gray, dashed] (0, 0) .. controls (.15, .02) .. (c) .. controls (.65, .26) .. (1, 0);
	
	% A scope to rotate and shift the boat
	\begin{scope}[rotate = \boatAngle, shift = {(c)}]
		% Boat outline
		\draw (.2, 0) .. controls (.15, -.05) .. (.1, -.05) -- (-.1, -.05) -- (-.1, .05) -- (.1, .05) .. controls (.15, .05) .. (.2, 0);
		
		% Boat direction vector
		\draw[->] (0, 0) -- (\vectorLength, 0);
		
		% v label
		\begin{scope}[shift={(\vectorLength/2, 0)}]
			\begin{scope}[scale=\textSize, transform shape]
				\node[anchor=south, font=\fontsize{10}{3}] at (0, 0) {$ v $};
			\end{scope}
		\end{scope}
		
		% Base line (for boat direction)
		\draw (0, 0) -- (-\boatAngle:\vectorLength);
		
		% Boat direction arc
		\draw (0:\vectorLength / 3 * 2) arc (0:-\boatAngle:\vectorLength / 3 * 2);
		
		% Boat center dot
		\draw[fill] (0, 0) circle (.01);
	\end{scope}
	
	% u label
	\begin{scope}[shift={($ (c) + (\boatAngle/2:\vectorLength/3*2) $)}]
		\begin{scope}[scale=\textSize, transform shape, rotate=\boatAngle/2]
			\node[anchor=west, font=\fontsize{10}{3}] at (0, 0) {$ u $};
		\end{scope}
	\end{scope}
	
\end{tikzpicture}
	\caption{De situatie}\label{fig:rivierplaatje}
\end{wrapfigure}

In dit verslag zullen wij een boot behandelen die met snelheid \( v \) ten opzichte van het water, tracht een optimaal pad naar de overkant van een rivier te vinden. 
De co\"ordinaat parallel aan de oevers noemen we \( x_2 \). De co\"ordinaat loodrecht op de oevers noemen we \( x_1 \). 
De hoek tussen de vaarrichting van de boot en de $ x_1 $-richting wordt $ u $ genoemd. Verder is voor iedere $ x_1 $ in de rivier, de stroomsnelheid van de rivier in de $ x_2 $-richting gedefni\"eerd als $ S(x_1) $. 
In figuur \ref{fig:rivierplaatje} is $ S(x_1) $ dus overal negatief. 
Tot slot hanteren we de notaties $ x_1^\prime(t) $ en $ x_2^\prime(t) $ voor de snelheden van de boot ten opzichte van de oevers op tijdstip $ t $ in respectievelijk de $ x_1 $- en $ x_2 $-richting.