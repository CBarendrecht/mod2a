\chapter{Discussie}
Ons onderzoek kenmerkte zich door de wiskundige focus die we gebruikten om het probleem te modelleren. We losten de problemen eerst grotendeels analytisch op om de rekentijd die de computer nodig zou had te minimaliseren. Met deze aanpak is het mogelijk om relatief complexe problemen in een korte tijd door te rekenen. De precisie van dit rekenwerk is zeer hoog.

Om dit onderzoek grotendeels analytisch te kunnen doen kent ons model een aantal grote versimpelingen ten opzichte van de werkelijkheid. Zo werken wij met een enkele wrijvingsterm welke afhankt van de snelheid om alle wrijvingsfactoren samen te vatten. Ook gaan we er van uit dat de boot in zeer korte tijd scherpe hoeken kan draaien, een onbeperkte hoeveelheid brandstof bij zich draagt en dat het gewicht van de boot niet verandert ondanks dat er brandstof verbruikt wordt. Op de boot na gaan we uit van een volledig constant systeem en de stroming varieert enkel in de \(x_1\) richting.

Met meer tijd hadden we graag de code voor het vinden van een numerieke oplossing voor de goedkoopste route afgerond. Het zou ook interresant zijn om het model uit te breiden naar stromingsprofielen welke niet alleen in de \(x_1\) richting veranderen maar ook in de \(x_2\) richting. 